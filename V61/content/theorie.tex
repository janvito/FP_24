\section{Theorie}\label{sec:theorie}
Nachdem zunächst auf den Grundlegenden Aufbau eines Lasers und die Einzelheiten des Entstehungsprozesses von Laserstrahlung und relevante charakteristische Eigenschaften der Wellen eingegangen wird, folgt mit diesem Wissen eine Einführung in die Funktionsweise des verwendeten \ce{He}-\ce{Ne}-Lasers.
\subsection{Aufbau eines Lasers}
Im wesentlichen besteht ein Laser aus drei Komponenten: dem \textbf{aktiven Medium}, der (selektiven) \textbf{Energiepumpe} und dem \textbf{Resonator}.\\
Das Aktive Medium ist ein Material, welches unter speziellen Vorraussetzungen die Fähigkeit besitzt, die Intensität von durchlaufendem Licht zu verstärken. Dies geschieht, da Atome in solchen Konfigurationen angeregt werden, die induzierte Emission von Photonen wahrscheinlicher als Absorption für bestimmte Frequenzen werden lassen(siehe \hyperref[subsec:entstehung]{Entstehung von Laserstrahlung}).\\
Die Energiepumpe liefert die nötige Energie um die Atome in die gewünschten angeregten Zustände zu heben. Erst durch sie kann der Laserbetrieb ermöglicht werden.\\
Der Resonator, typischerweise bestehend aus zwei Spiegeln, sorgt dafür, dass das Licht mehrmals das aktive Medium durchläuft indem er es hin- und herreflektiert. So kann die Verstärkung der Intensität signifikant genutzt werden, um einen stabilen Strahl zu konstruieren. Diese Energie des Laserstrahls wird zu einem großen Teil im Resonator in wenigen Resonatormoden gespeichert, was zu einer hohen Strahlungsdichte in ausgewählten Wellenlängen führt.\cite{Demtroeder} 
\subsection{Entstehung von Laserstrahlung}\label{subsec:entstehung}

\subsection{Eigenschaften von Laserstrahlung}
\subsection{Funktionsweise eines He-NeLasers}
\section{Durchführung}\label{sec:durchfuehrung}
Zunächst wird der Justageprozess dokumentiert, da dieser mehrfach Verwendung findet, anschließend werden die Verschiedenen Messmethoden erläutert.
\subsection{Justageprozess}
Sowohl vor Inbetriebnahme des Hauptlasers, als auch nach jeder Verschiebung der Resonatorspiegel müssen die Spiegel sowie gegebenenfalls das Laserrohr mittels der Justierschrauben eingestellt werden, damit die höchstmögliche Intensität udn somit die größte Stabilität des Strahls erreicht wird.
Nach Einschalten des Justagelasers werden sowohl Laserrohr und die nacheinander eingesetzten Resonatorspiegel so justiert, dass sowohl der ausgekoppelte Strahl, als auch der Rückreflex, bestmöglich die Mitte der Beugungsblenden an beiden Enden der Schiene trifft und konzentrische Beugungsringe aufweist.
Anschließend kann das Hauptlaserrohr mit einem Strom $I=\SI{6.5}{\milli\ampere}$ eingeschaltet werden und es setzt unter genauer Nachjustierung die Lasertätigkeit ein. Nun kann der Justierlaser wieder ausgeschaltet werden und die Intensität unter Verwendung einer Photodiode und den Justierschrauben maximiert werden.
\subsection{Messmethoden}
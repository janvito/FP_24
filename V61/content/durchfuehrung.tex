\section{Durchführung}\label{sec:durchfuehrung}
Zunächst wird der Justageprozess dokumentiert, da dieser mehrfach Verwendung findet, anschließend werden die Verschiedenen Messmethoden erläutert.
\subsection{Justageprozess}
Sowohl vor Inbetriebnahme des Hauptlasers, als auch nach jeder Verschiebung der Resonatorspiegel müssen die Spiegel sowie gegebenenfalls das Laserrohr mittels der Justierschrauben eingestellt werden, damit die höchstmögliche Intensität udn somit die größte Stabilität des Strahls erreicht wird.
Nach Einschalten des Justagelasers werden sowohl Laserrohr und die nacheinander eingesetzten Resonatorspiegel so justiert, dass sowohl der ausgekoppelte Strahl, als auch der Rückreflex, bestmöglich die Mitte der Beugungsblenden an beiden Enden der Schiene trifft und konzentrische Beugungsringe aufweist.
Anschließend kann das Hauptlaserrohr mit einem Strom $I=\SI{6.5}{\milli\ampere}$ eingeschaltet werden und es setzt unter genauer Nachjustierung die Lasertätigkeit ein. Nun kann der Justierlaser wieder ausgeschaltet und die Intensität unter Verwendung einer Photodiode und den Justierschrauben maximiert werden.
\subsection{Messmethoden}
Es werden fünf verschiedene Messvorgänge zur Bestimmung der Lasereigenschaften durchgeführt.
\subsubsection{Stabilitätsbedingungen}
Wie in \autoref{subsec:eigenschaften} erläutert, ist die Stabilität eines jeden Lasers abhängig von den Krümmungsradien der Spiegel und dem Abstand dieser. Es werden Spiegelabstände unter laufender Nachjustierung maximiert bis kein stabiler Strahl mehr realisiert werden kann. Diese Messung wird für zwei unterschiedliche Spiegelkonfigurationen durchgeführt, es werden nacheinander die zwei konkaven High-Reflectivity Spiegel aus \autoref{tab:spiegel} verwendet.
\subsubsection{TEM-Moden}
Indem ein dünner Wolframdraht mit Durchmesser $d=\SI{5}{\micro\meter}$ zwischen Laserrohr und High-Reflectivity Spiegel platziert, justiert und verschoben wird, lassen sich verschiedene Moden im Strahl stabilisieren. Nach der optischen Erfassung einer Mode auf einem Schirm kann mittels einer senkrecht zur Strahlrichtung verschiebbaren Photodiode mit kleinem Wirkungsfenster die Intensitätsverteilung der Mode gemessen werden.
\subsubsection{Polarisation}
Ein zu Verfügung stehender Polarisator wird zwischen dem Auskoppelspiegel und einer Photodiode auf die Schiene gestellt und die Intensität des Strahls unter variation des Polarisationswinkels gemessen.
\subsubsection{Multimoden}
Mit einer hochfrequenten Photodiode einer Bandbreite von bis zu \SI{1}{\giga\hertz} und einem geeigneten Oszilloskop wird die Strahlung auf ihr Fourierspektrum untersucht. Dies wird für verschiedene Resonatorlängen wiederholt.
\subsubsection{Wellenlänge}
Es wird ein Gitter vor einen Schirm am Ende des Lasers gesetzt und sowhol der Abstand der Beugungsmaxima als auch die Distanz zwischen Gitter und Schirm gemessen. Dies wird für zwei verschiedene Gitterkonstanten wiederholt.